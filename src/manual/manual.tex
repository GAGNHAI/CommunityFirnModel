\documentclass{article}

\usepackage{hyperref}

\begin{document}

\title{Community Firn Model (CFM)}
\author{Jessica Lundin, Max Stevens, Paul Harris\\ Earth and Space Sciences Department, University of Washington}

\maketitle
\newpage

\section{Download Required Software}


\begin{itemize}
  \item Download individually
  \begin{enumerate}
    \item Download Python at \href{http://www.python.org/getit/}{www.python.org/getit/}.

    Choose version 2.7.x as it is more stable than version 3.3.x.

    For Mac users, MacPorts is a useful way to maintain software that includes Python. Download MacPorts at \href{http://www.macports.org/install.php}{www.macports.org/install.php}.
    
    \item Download NumPy and SciPy using the Anaconda distribution at \href{http://continuum.io/downloads}{continuum.io/downloads}.

    More detailed instructions for installation can be found at \href{http://docs.continuum.io/anaconda/install.html}{docs.continuum.io/anaconda/install.html}.
    
    \item Download matplotlib at \href{http://matplotlib.org/downloads.html}{matplotlib.org/downloads.html}.

  \end{enumerate}
  \item Download everything (and more) in one package
  
  \begin{enumerate}
    \item Enthought is a Python distribution that includes Python, NumPy, SciPy, and matplotlib, as well as a GUI, IPython console, and other features. Download Enthought at \href{https://www.enthought.com/store/}{www.enthought.com/store/}.

    The free download contains everything we need.
  \end{enumerate}
\end{itemize}

\section{Download the firn model}

\begin{itemize}
  \item Directly from the website
  \begin{enumerate}
    \item Windows: (link to zip file).
    \item Mac OS/UNIX/LINUX: (link to tar ball).
  \end{enumerate}

  \item Using the GitHub repository 
  \begin{enumerate}
    \item Find our GitHub repository at \href{https://github.com/jessicalundin/FMbeta/}{github.com/jessicalundin/FMbeta/}.
    \item Clone in Desktop or Download ZIP.
    \item You should now have a folder ``FMbeta'' that contains the whole project.
  \end{enumerate}
\end{itemize}

\section{Files and Folders in ``FMbeta''}
\begin{itemize}
  \item code

  The ``code'' folder contains the files needed to run the model.
  \begin{enumerate}
    \item firnmodel.py is the main model code.
    \item config.json is a list of variables that are used by firnmodel.py (This is what you will change.)
    \item plot.py produces plots from data files created by firnmodel.py.
  \end{enumerate}
  \item manual

  The ``manual'' folder contains a manual with instructions, variable descriptions and other information.
  \begin{enumerate} 
    \item manual.tex is the LaTeX document containing the manual for the model.
    \item manual.pdf is the PDF of the manual produced by manual.tex.
    \item Any other files are produced by manual.tex and are generally unimportant.
    
  \end{enumerate}
\end{itemize}

\section{How to Run an Experiment}

The model runs in python. Running the model from the command line is simplest. Make sure that the python that is called from the command line is the python installation that includes numpy and scipy modules. First you must configure the config.json file (can be named whatever you want). Then you must run a spin up, and then you run the main model run. 

\subsection{.json file setup}

The .json file is called by the model to find numerous parameters. This is where you can easily specify the temperature, accumulation rate, etc. 

Note from Max on June 18, 2014: we need to get a section of this manual to describe the variables, e.g. is the 'bdot0' in units of ice equivalent or water equivalent? (I am about 80\% sure it is ice).

\subsection{Running the model}
From the command line, type:

\begin{verbatim}
python firnmodel.py config.json -s
\end{verbatim}

to run the spin up. The -s is the option to run the spin up. The spin-up output will be stored in the results folder that you specify in the config.json file that you are calling. Then, to run the main model run, type:

\begin{verbatim}
python firnmodel.py config.json
\end{verbatim}

If you are in ipython, you use the same commands except replace ``python'' with ``run'' at the start. 

\section{Output Files and Plots}
\begin{itemize}
  \item Data Files

  Each output file consists of comma seperated values. The first element in each row represents the time step in years.
  The remaining elements in the row represent the value of the chosen element of the firn at each point along the grid.
  These output files are:
  \begin{enumerate}
    \item age.csv

    How old the firn is in  \begin{math}(years)\end{math}.
    \item density.csv

    The density of the firn in \begin{math}(kg \: m^{-3})\end{math}
    \item depth.csv

    The depth of the firn in  \begin{math}(m)\end{math}

    \item temp.csv

    The temperature of the firn in  \begin{math}(K)\end{math}

    \item r2.csv (only if grain growth is turned on)
    
    \item 
    
  \end{enumerate}
  \item Plots

  Plots of output data are produced by setting ``plotting'' to ``on'' in config.json. Each plot consists of the inital and final
  profiles of the data plotted against depth.
\end{itemize}


\newpage
\appendix
\section{Model Variable Appendix}

\begin{tabular}{ |l| p{5cm} | p{3cm} |}
  \hline
  Variable &
  Description &
  Options \\
  \hline
  BCtemp &
  -- &
  \\
  \hline
 BCbdot &
  --  &
  \\
  \hline
   BCrho &
  -- &
  \\
  \hline
   physRho &
  -- &
  \\
  \hline
   physGrain &
  Turns grain physics calculations on or off. &
  ``on'', ``off''\\
\hline
   heatDiff &
  Turns heat diffusion calculations on or off. &
  ``on'', ``off''\\
  \hline
   Ts0 &
  Beginning Surface Temperature &
   \\
  \hline
   rhos0 &
  Beginning Surface Density &
  \\
  \hline
   bdot0 &
  Beginning Accumulation Rate &
  \\
  \hline
   r2s0 &
  Beginning \begin{math} r^2 \end{math} &
  \\
  \hline
   years &
  Number of years to run the experiment. &
  \\
  \hline
   stpsPerYear &
  Number of steps for each year the experiment is run. &
  \\
  \hline
  gridH &
  Grid Height &
  \\
  \hline
   gridbase &
  -- &
  \\
  \hline
  gridlen &
  Grid Length &
  \\
  \hline
  sPerYear &
  Seconds Per Year &
  \\
  \hline
  rhoi &
  Density of ice &
  \\
  \hline
  rhoiMgm &
  Density of ice in (\begin{math} Mg \: m^{-3} \end{math}) &
  \\
  \hline
   rho1 &
  -- &
  \\
  \hline
   rho2 &
  -- &
  \\
  \hline
   Q1 &
  -- &
  \\
  \hline
   Q2 &
  -- &
  \\
  \hline
  k1 &
  -- &
  \\
  \hline
   k2 &
  -- & 
 \\
  \hline
   a &
  -- &
  \\
  \hline
   b &
  -- &
  \\
  \hline
   R &
  -- &
  \\
  \hline
   g &
  Acceleration of Gravity &
  \\
  \hline
   H &
  -- &
  \\
  \hline
  kg &
  Grain Growth Constant &
  \\
  \hline
  Eg &
  Grain Growth Activation Energy &
  \\
  \hline
\end{tabular}
\end{document}